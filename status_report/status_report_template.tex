    
\documentclass[11pt]{article}
\usepackage{times}
    \usepackage{fullpage}
    
    \title{Full abstraction for the encoding of session types}
    \author{Iain McLean - 2308499m}

    \begin{document}
    \maketitle
    
    
     

\section{Status report}

\subsection{Proposal}\label{proposal}

\subsubsection{Motivation}\label{motivation}

\emph{In the paper Session Types Revisited, a simple session-typed Pi-Calculus was defined with a secure encoding system to a standard typed pi-calculus. The point of this paper is to extend the work that was completed in Session Types Revisited and prove that the observable behaviour of the session-
typed calculus and the linearly typed pi-calculus is fully abstract}

\subsubsection{Aims}\label{aims}

\emph{Extend $\pi^{ses}$ (Session Typed Pi Calculus) and $\pi^{lin}$ (Standard Typed Pi Calculus) such that there is barbed congruence so that processes' observational behaviour can be tested using may must congruence relations. Then prove there is some relation that satisfies may must testing to prove some process in $\pi^{ses}$ and $\pi^{lin}$ are observationally equivalent encoded in any context.}

\subsection{Progress}\label{progress}

\begin{itemize}
    \setlength{\itemsep}{0pt}
    \setlength{\parskip}{0pt}
\item Read over Session Types Revisited
\item Read over the Full Abstraction Paper (short and long)
\item Read over Types and full abstraction for polyadic Pi-calculus
\item Read over Testing Theories for Asynchronous Languages
\item Read over The Pi-Calculus: A Theory of Mobile Processes
\item Defined a labelled transition system for Pi-Calculus with Session types
\item Defined Barbs and Barbed bisimulation for the Pi-Calculus with Session types and the linearly typed Pi-Calculus
\item Defined a maximal reduction sequence (for the proof in the Full Abs paper)
\item Started on the Full Abstraction Proof

\end{itemize}

\subsection{Problems and risks}\label{problems-and-risks}

\subsubsection{Problems}\label{problems}

\begin{itemize}
    \setlength{\itemsep}{0pt}
    \setlength{\parskip}{0pt}
\item Getting to grips with theoretical computer science
\item Understanding the mathematical terms surrounding this topic
\item May be spending too much time reading
\end{itemize}


\subsubsection{Risks}\label{risks}

\begin{itemize}
    \setlength{\itemsep}{0pt}
    \setlength{\parskip}{0pt}
\item Getting sidetracked by other reading
\item Theory of Mobile Processes book follows a different more complex proof for Full Abstraction which may need to be studied further which may delay me further.
\end{itemize}

\subsection{Plan}\label{plan}

\begin{itemize}
    \setlength{\itemsep}{0pt}
    \setlength{\parskip}{0pt}
\item Week 1 - 2: More in depth draft of results and proposed extensions
\item Week 3 - 5: Refine definition, begin write up of dissertation
\item Week 6 - 8: Write up draft of dissertation
\item Week 8 - 10: Final write up
\end{itemize}

    
\subsection{Ethics and data}\label{ethics}

Options for ethics: \\
This project does not involve human subjects or data. No approval required.


\end{document}
